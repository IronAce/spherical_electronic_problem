\documentclass[a4paper,12pt]{article}
\usepackage{fancyhdr}
\usepackage{fixltx2e}
\usepackage{amsmath}
\usepackage{graphicx}
\usepackage{braket}
\usepackage{amssymb}
\begin{document}
\numberwithin{equation}{section}
\numberwithin{figure}{section}
\pagestyle{fancy}
\fancyhead[L]{\fontsize{9}{9} \selectfont Jonathan Lahnsteiner}
\fancyhead[R]{\fontsize{9}{9} \selectfont 15.01.2015}

\title{Associated Legendre Polynomials}
\maketitle

\section{General}

Assocoiated Legendre Polynomials are the canonical solutions to the General Legendre differential equation \ref{1}. The term canonical may refer in this context
to a natural representation. For example canonical coordinates are a set of coordinates which describe a system properly at any given time. But canonical
may also refer to a unique representation of a given mathematical problem like for example the solutions to the generalized Legendre differential equation.

\begin{equation}
 \frac{d}{dx} \left [ \left ( 1-x^{2} \right )\frac{d}{dx}P_{l}^{m} \left ( x \right ) \right ] + \left [ l\left( l + 1 \right ) 
   - \frac{m^{2}}{1-x^{2}} \right ]P_{l}^{m} \left ( x \right ) = 0
   \label{1}
\end{equation}

\noindent
In this equation $P_{l}^{m} \left ( x \right )$ denotes the associated Legendre polynomial and m and l are integers which refer to the degree and order
of the considered polynomial respectively. The equation has a finite number of solutions in the intervall $[-1:1]$ only if $0\leq m\leq l$.\\
If m is even $P_{l}^{m} \left ( x \right )$ is a polynomial. If $m=0$ and $l \in N $ then one receives the Legendre Polynomials. 
Moreovoer it should be mentioned that the Legendre polynomials play an mportant role when determining the so called spherical harmonics.

\section{Definition}








\end{document}